In the worst case, we insert the smallest number into tree, so we need to swap from the leaf to the root in \textit{bubble-up} way. That results in $\lfloor \log(n) \rfloor $.
In the worst case after the delete, we replace the root with the largest value then we need to swap from the root to the leaf in \textit{bubble-down} way. That results in $\lfloor \log(n) \rfloor$.

For amortized analysis, we can sum depths of all nodes as a potential function. Therefore, since tree is balanced, we sum \textit{logarithms} of all nodes and we have the following for $n$ items of a heap:

  $$
    \Phi = \sum_{i=2}^{n} \log i
  $$

\textit{insert} operation has a logarithmic amortized cost:

  $$
    amortized\text{ }cost = actual\text{ }cost + \Delta\Phi = \log(n) + \log(n+1) \equiv O(\log n)
  $$
  
However, \textit{delete} operation has better amortized cost:

  $$
    amortized\text{ }cost = actual\text{ }cost + \Delta\Phi = \log(n) - \log(n) \equiv O(1)
  $$