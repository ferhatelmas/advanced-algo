Stack is capable of inserting and deleting at the end of a list but queue accepts elements from one end and elements are removed from the other end so they have quite different operational semantics. However, a queue can easily be simulated by two stacks, namely \textit{inbox}, $S_{i}$ and \textit{outbox}, $S_{o}$.

We will use $S_{i}$ for Insert-Queue and $S_{o}$ for Delete-Queue operation. Therefore, we push the element into $S_{i}$ if we want to insert a new element into queue. Delete operation contains two scenarios according to state of $S_{o}$. If it isn't empty, we pop its top element; otherwise, we transfer all elements from $S_{i}$ to $S_{o}$ respectively by popping from $S_{i}$ and pushing into $S_{o}$. Then, we have the required element at the top of $S_{o}$ so we just pop it.

For analysis, we define potential function: $\Phi = 2n_{i}$, where $n_{i}$ is the number of elements in stack $S_{i}$. There is a multiplier 2 since when \textit{outbox} is empty, we transfer elements from \textit{inbox} by pop and push, where each one has a cost of 1 unit, totally 2. Then, we have the following amortized cost for each Insert-Queue operation:

$$
	actual\text{ }cost + \Delta\Phi = stack\text{ }push + change\text{ }of\text{ }inbox = 1 + 2 \leq O(1)
$$

For each Delete-Queue operation, we have to consider two cases. First, easier one, when $S_{o}$ is not empty, the amortized cost is:

$$
	actual cost + \Delta\Phi = stack\text{ }pop + change\text{ }of\text{ }inbox = 1 + 0 \leq O(1)
$$

When $S_{o}$ is empty, the amortized cost is:

\begin{align*}
	actual\text{ }cost + \Delta\Phi &=  n\text{ }stack\text{ }pop + (n-1)\text{ }stack\text{ }push + change\text{ }of\text{ }inbox \\
	&= (2n_{i} - 1) + (0 - 2n_{i}) \leq O(1)
\end{align*}

As a result, amortized cost of each Insert-Queue and Delete-Queue operations is constant.