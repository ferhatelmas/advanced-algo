Fitstring representation are \textbf{not} unique because "011" can always be written as a carry in the form of "100". Moreover, even if we prefer one over another and use it whenever possible, some numbers can be written in multiple forms.

$$
1(0)\{2n+1, 2n+1\} = 1(01)\{n, n\}
$$ 

In the former representation "1" is followed by $2n + 1$ "0" and in the latter, "1" is followed $n$ "01". 

\vspace{0.3cm}
$increase\text{ }operation$: \\
First, how a carry is created:
\begin{itemize} 
\itemsep -2pt
\item If last bit is "0", it is just flipped to "1".
\begin{itemize}
\item $a0 \rightarrow a1$
\end{itemize}
\item If last bit is "1", the previous bit is considered
\begin{itemize}
\itemsep -2pt
\item if it is "0", the fitstring ends with "01" and this "01" is flipped to "11". Unlike binary, this works because last two bits represent value of 1.
\begin{itemize}
\item $a01 \rightarrow a11$
\end{itemize}
\item if is is also "1", the fitstring ends with "11" and this "11" is flipped to "01". Then, a carry is created and propagated to more significant bits.
\begin{itemize}
\item $a11 \rightarrow (a+1)01$
\end{itemize}

\end{itemize}
\end{itemize}

Secondly, general case at the position of $i$ with a carry:
\begin{itemize}
\item If it is "0", it is just flipped to "1".
\begin{itemize}
\item $a(0+1)b \rightarrow a1b$ 
\end{itemize}
\item If it is "1", the number increased to $2F_{i}$ and carry must be propagated. By using the property of fibonacci numbers, $2F_{i} = F_{i} + F_{i-2}$, carry can be propagated one up and two down. Then, following rules are defined:
\begin{itemize}
\item $0(1+1)00  \rightarrow 1001$ 
\item $0(1+1)01a \rightarrow 1010(a+1)$ 
\item $1(1+1)00  \rightarrow (1+1)001$ 
\item $1(1+1)01a \rightarrow (1+1)010(a+1)$ 
\end{itemize}
\end{itemize}

As a result of these rules, the cost depends on propagation of the carry. However, each increment leaves alternating "0" and "1"s and at most one pair of adjacent "1".

\clearpage

$decrement\text{ }operation$: \\
\vspace{-0.3cm}
\begin{itemize}
\item If last bit is "1", it is flipped to "0"
\begin{itemize}
\item $a0 \rightarrow a1$
\end{itemize}
\item If last is "0" and previos bit is "1", then fitstring ends with "10", so "10" is flipped to "00" since last two bits represent value of 1.
\begin{itemize}
\item $a10 \rightarrow a00$
\end{itemize}
\item If last two bits are "00" but third bit is "1", then fitstring ends with "100", so "100" is flipped to "010".
\begin{itemize}
\item $a100 \rightarrow a010$
\end{itemize}
\item Otherwise, we need a borrow from more significant bits.
\begin{itemize}
\item Search least significant "1" in the fitstring and assume it is found at position $i$.
\item As explained above, this fit should have two zeros to its right. Therefore, "100" is flipped to "011".
\item We have a "1" fit at the position $i-2$ and if it has two zeros to its right, recursively the previous step is also applied this fit.
\item At the end, we hit the base cases and do the actual decrement:
\begin{itemize}
\item $100000000\dots00 \rightarrow 10101010\dots10$
\item $100000000\dots0 \rightarrow 10101010\dots0$
\end{itemize}
\item Cost is depends on the number of consecutive 0s but leaves alternating 0 and 1s and decrement can create at most one pair adjacent zero pair.
\end{itemize}
\end{itemize}

Increment and decrement are costly in terms of trailing 1s and 0s, respectively but they transform the fitstring into alternation and have the next operation taken constant time.

Since cost depends on consecutive fits, we need to count them where alternation is the perfect case. Our potential function keeps track of the number of adjacent same fits:

$$
\Phi(f) = \Phi(f_{n}f_{n-1}f_{n-2} \dots f_{1}f_{0}) = \sum_{i=1}^{n} f_{i-1} = f_{i}
$$

If an operation has a real cost of $c$, then it must have $c$ identical trailing fits. Moreover, it should have arrived at least the potential of $c-1$ so that it contains $c$  identical fits but operation decreases potential by $c-1$ via transforming string into an alternation. However, each operation can create one pair identical fits in doing alternation but this cost is dominated by constant. As a result, the sum of total cost and potential change is bounded by a constant.