Suppose that we have a directed network that contains one good node $g$ and various bad nodes $b_i$. We consider $g$ to be the source of the network and then we use a virtual node as a sink and we connect all nodes with out-degree 0 to this virtual node, sink.

Starting from the sink, we traverse the graph to search good node in the reverse direction to set comparative costs to edges while applying the following rule:
\begin{itemize}
  \item Set the sink to 0.
  \item If the visited node is a normal node (neither good nor bad), the comparative cost of the edge is the total out-flow of the node. $c_{ij} = \sum \textit{out-flow}_j$
  \item If the visited node is a bad node, the comparative cost of the edge is the total out-flow of the node and 1. $c_{ij} = 1 + \sum \textit{out-flow}_j$
  \item If the visited node is a good node $g$, this part of the algorithms finishes.
\end{itemize}

After getting comparative costs of the edges, we multiply these values with a huge value to prevent the flow from going over bad nodes. This leads us to min-cost max-flow problem and we try to send the flow $d$ from the good node to our virtual node where $d$ is the out-degree of the good node. This can be accomplished in $O(V \cdot E^2) \leq O(V^5)$ complexity by tweaking \textit{Ford-Fulkerson(Edmonds-Karp)} with a guaranteed termination. After the algorithm completes, we find the edge with the highest cost that's not included into the path found by \textit{Ford-Fulkerson} and also has a bad node as an end point. This can easily be done with breadth-first search starting from the good node $g$ so complexity is bounded by $O(E)$, but in practice, it will be much lower. When the edge with the given constraints is found, we may remove it according to improvement in $|f(S)| - a \cdot |S|$. The size of $S$ will increase by 1 since we will remove one edge, and to find the change in $|f(S)|$, we can do a search for the virtual node again starting from the good node $g$ without using preselected edge(assuming it is removed). If it improves, we really remove preselected edge and continue iterate while applying first cost generation and then min-cost max-flow to the graph that has one less edge. Otherwise, we return the current value of the variant.

For the proof, at each iteration, we find the edge that connects the good node $g$ to the maximum number of bad nodes. We only remove this edge so the number of the edges to be removed is in its theoretical limit. Moreover, we disconnect the maximum number of bad nodes. Therefore, by minimizing removed edges and maximizing disconnected bad nodes, maximum of the function will be found by the above algorithm.