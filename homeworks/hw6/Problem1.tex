In the model, we are trying to match phones and base stations, so this problem can be reduced to well-established Hungarian algorithm with some additional constraints.

Here, we can think base stations and phones as a bipartite graph because there are no connections within base stations or phones. Moreover, each phone must be assigned to a distinct base stations. That's why we are searching for one-to-one match. We will use Hungarian algorithm as a black box to find this matching but it can also be solved by Ford-Fulkerson if we add virtual source and sink nodes and think the problem as a min-cost max-flow problem.

Another constraint is distance limitation because if a phone is further away from its assigned base station than a given constant $c$, this assignment isn't valid since phone can not utilize of base station service due to the distance. Therefore, we have a threshold cost model. If a phone is far away from a base station more than given $c$, then the cost is set to very high value. Otherwise, cost is basically zero because we don't differentiate the cost under distance $c$.

Initial setup requires us to calculate the distance of each phone to each base station. This can easily be done via two nested for-loops which is $O(n^2)$ complexity. This is fine since it will be dominated by Hungarian algorithm which has a higher complexity. While calculating distance, we also create a cost matrix in the threshold model, 0 or $+\infty$.  Cost matrix, in addition to distance calculation, can only increase the multiplier in front of $n^2$ so complexity is same, $O(n^2)$.

After calculations of distances, we run Hungarian which has a complexity of $O(n^3)$ for $n$ phones and $n$ base stations. If we can get $n$ assignment, we start to driving. Otherwise, \textit{fully connectedness} can not be satisfied so we immediately return false. Throughout the driving, for each unit of driving, we check the distance between the moving phone and its assigned base station. If the distance is below $c$, there is nothing to do and we continue with next unit of travelling. However, if distance is more than $c$, we need to change the assignment of the moving phone since it can't be serviced by its previous base station any more. In the cost matrix, we need to update the distance of moving phone to other base stations which can accomplished in $O(n)$. Then, we again run Hungarian. If it can find a $n$ assignment, we can go further, otherwise we return false. This procedure is repeated $k$ times for $k$ units of distance. Finally, we return true.

In short, 
$$
  \textit{Total Cost} = \textit{Cost Setup} + k \cdot Move = O(n^2) + k \cdot O(n^3) \leq O(n^3)
$$

\[
  Move = \left\{ 
  \begin{array}{l l}
    O(1) & \quad \text{if distance $d \leq c$}\\
    O(1) + O(n) + O(n^3) & \quad \text{if distance $d > c$}\\
  \end{array} \right.
\]