To find a minimum triangulation of a convex polygon, we have to first make some general statements and observations. First, it has to be noted that for every possible triangulation of a polygon, there is a triangle that if removed from the polygon it splits it into two distinct polygons. Moreover, these two polygons are also validly triangulated (or are triangles). For a minimum triangulation (one where the edges of the triangles are the shortest possible), if we split the polygon into two as described above, the triangulation of each of the two triangles must also be minimum. Also, for a polygon with 4 edges, the optimal triangulation would be the one with the smallest diagonal. We can also notice that in every triangulation, each edge corresponds to exactly one triangle while each inner edge corresponds to exactly two triangles. Thus the cost of the triangulation would be the sum of the perimeter of each triangle decrease by the sum of the edges of the polygon and divided by two. However, since the sum of the edges of the polygon is constant, minimising the sum of the perimeters of each triangle, would correspond to minimising our triangulation. 

Thus we can device a dynamic algorithm that will start splitting the polygon into smaller ones (by selecting a diagonal), trying to find an optimal triangulation of the smaller triangles. In the initial polygon there are $\binom n 2$ chords that we could start from. In order to avoid recalculating costs when same polygons occur, we are going to use memoisation and store the minimal triangulation (and its cost) of each smaller polygon the first time we calculate it so that we can have access to it every time it is needed. Thus the algorithm would be the following:

Consider that since we have the list of its vertices, in counterclockwise order along the perimeter, we can consider a sub-polygon that is defined as $Sub(i,j)$ starting at vertex $i$ and ending at vertex $j$ containing all other vertices in counterclockwise order. As such, we initialize matrix $SP$, which will contain cost of triangulations of each sub-polygon, with zeros. $SP(1,n)$ will contain cost for polygon starting at vertex $1$ and ending at vertex $n$, this is what we are looking for. Fill $SP$ as follows: $SP(i,k) = min_j \{SP(i,j) + SP(j,k) + perimeter(i,j,k)\}$ Where $k > i+1$, $j \in [i,k]$ and perimeter(i,j,k) is the perimeter of the triangle with points i,j and k.\\

\textbf{Running-time:}\\
The time to calculate the values of the matrix would be $O(n^2)$ (since there are $n^2$ elements in the matrix). Each element requires $O(n)$ operations as we need to iterate over j in order to test every possible triangulation and pick the minimum. Hence, the algorithm is $O(n^3)$.