Firstly, we draw attention onto some important characteristics of the problem:
\begin{itemize}
  \item Points must be traversed one by one because turning the direction (other than points) only increases the total delay.
  \item Visiting points is costless when we get there. Thus, if we are passing a non-visited node, we should note it as visited. As a result of this property, we can define an interval as visited. That's why next candidate point is the one right or left of the interval. 
\end{itemize}

We need differences between points so we create a matrix of $n \times n$ to store differences. This matrix is created in $O(n^2)$. After we get matrix, we can lookup differences in $O(1)$.

Since we could see visiting process as an interval, we want a $n-length$ interval. We define our function(states) as $delay(i, j, k)$ where it is the total cost at point $i$ with leftmost point $j$ and rightmost point $k$. Since in an interval all points are visited, visiting a new point can extend the interval. Therefore, $delay(i, j, k)$ seems to have $n^3$ states but $i$ actually can be equal to only $j$ or $k$ which means $i$ is just a binary flag. As a result, we have only $n^2$ states.

With defined $delay(i, j, k)$ function, we can easily devise the minimum cost of $n-length$ path by
$$
  \min(delay(1, 1, n), delay(n, 1, n))
$$ 
In each step, (with explained $i$) we will chop off  leftmost or rightmost point and then take their min. Base case of our recursive algorithm is the interval with length of 2 when we simply return the difference between two given points.

$$delay(i, j, k) = \left\{ 
  \begin{array}{l l}
    min( \\ \hspace{.2cm} dist(j, j+1) + 2 \cdot delay(j+1, j+1, k), \\ \hspace{.2cm} dist(k, k-1) + 2 \cdot delay(k-1, j, k-1)) & \quad \text{if } k-j+1 > 2 \\
    dist(j, k) & \quad \text{if } k-j+1 = 2
  \end{array} \right.
$$

2 in the above algorithm comes from the nature of the delay function: while visiting point $i$, all visiting costs prior to $i$ are added into delay of $i$.
$$
  \sum_{i=2}^{n}{\sum_{j=2}^{i}{dist(i, j)}}
$$

In analysis, our $delay$ function has $n^2$ states because $j$ and $k$ parameters can take any value from $[1, n]$, precisely $\sum_{j=1}^{n-1}{\sum_{k=j+1}^{n}{1}} \leq O(n^2)$. Moreover, we need to run algorithm for different starting points since we have $O(n)$ points and each run is $O(n^2)$, we get $O(n) \cdot O(n^2) \leq O(n^3)$ overall. Other costs such as finding actual path after getting minimum(optimal) delay or creating $n \times n$ matrix for differences will be dominated by $O(n^3)$.

Next, we will define the procedure to get the actual path since question asks us for it. We can modify the above algorithm (adding $O(1)$ time work) to make bookkeeping where it found the optimal. Therefore, we add predecessor pointer which is only the number of the point in the list. For $delay(i, j, k)$, if we came from $j+1$, we add $j+1$ for $i$. Otherwise, we add $k-1$. Therefore, following these pointers in the reverse order gives us the actual path.

One final remark is that we assumed that points are given in sorted order. However, it isn't a problem because we can sort by ourselves unless they are in sorted order and it will also be dominated.