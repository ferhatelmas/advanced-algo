We will present the construction of a randomized algorithm that examines points randomly chosen from our initial set of points in order to determine whether there exists a line dividing them in color sets. 

Initially we arbitrarily select three points. By definition, two points will be of one color. If the third one is of the same color, we put it back to the set of unchecked points and take another one until we come across a point of the other color. Since no three points are collinear we are guaranteed that there is at least one line that can separate the points color-wise. To design that line we can draw a line that unites the points of the same color. Then the separating line will be parallel to this line and in the middle of the distance between the different colors. Once we have defined our separating line, we start checking all points to see if they are in the correct side of the line. At the same time we are calculating their distance to the separating line and then always store the point closest to this line for each color set. When a point violates the separating line border, we attempt to redefine the line's position by rotating the line around the previously saved point of the minimum distance of the same color. After the rotation of the line we will have to check every other point previously added to see if it is on the correct side of the line, if any point is not we will return false otherwise we will continue by adding more points. If the algorithm runs to the end, i.e. when all points have been picked we return true.

Moving to the time analysis we make the following assumptions: Checking whether an added point is on the correct side of the line can be done in O(1) using the line equation and the coordinates of the point. Thus, for a point that does not lead to a conflict the running time is constant. However if we discover a conflict we first have to rotate the line which is O(1), but then we need to check all other points if a conflict is found this is O($k$), where $k$ is the number of points added thus far. Thus the worst case running time is if we need to do this check every time we add a point leading to a total running time of O($(n + m)2$). 

In order to calculate the expected running time we perform a reverse analysis, i.e. we start with the state obtained at the end of algorithm, and analyse the cost of removing a point at random or more precisely what was the work in the actual running of the algorithm to add the point we just removed. The removed point can be one of two cases: \begin{enumerate}
	\item[(a)] it is a point which does not lie on the line, i.e. the point was added
to the correct side of the line with any adjustment of the line (O(1)).
\item[(b)] the point lies on the line, which meant we had to do the expensive part of the algorithm to rotate the line to its position and check all other points if they were on the same side, this is O(k).
\end{enumerate}
However the probability of choosing a case (b) point is only $\frac{2}{k}$ since there are $\binom{k}{2}$ ways of choosing 2 points out of $k$ points. Thus, the probability of choosing a point of (a) is consequently $1-\text{case } (b) = 1 - \frac{2}{k}$. The total expected running time when removing all points, based on the above calculations would be : $$\sum^{n+m}_{k=3}{\frac{2}{k}O(k)+\frac{k-2}{k}O(1)}\leq\sum^{n+m}_{k=3}{\frac{2}{1}O(k)+1O(1)}=(n+m)O(1)\in O(n+m)$$