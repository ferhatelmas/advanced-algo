  
  \begin{itemize}
    \item Max potential(Proof by contradiction):
      Suppose that there is another binary tree with a greater potential than the linked list but with the same $n$ number of nodes. It would mean that at some height of the tree there is one node having more nodes in its subtree than the equivalent node in the linked list. However this would be impossible since in a linked list, the only nodes of a tree that are not included in a node's subtree are its (grand)parent node(s). Therefore, assumption must be wrong and linked list has the maximum potential. Moreover, the same conclusion can be drawn by construction. Tree has $n$ elements so $w(root)$ is $n$ and child node can at most have $n-1$ nodes in its subtree and following its child can at most have $n-2$ nodes. This ends up in linked list structure.
    \item Min potential:
      Following the same reasoning in max potential, if a parent node divides its subtrees equally between children then children nodes can have the minimum potential because for potential we take the logarithms of the number of nodes and it depends on the height of the tree where summation of number of nodes in two children doesn't matter. Parent node has $m$ nodes, first child has $m_1$ and so second child has $m-m_1$ nodes:
      
      $$
        min \left\lbrace \log(m_1) + \log(m-m_1) \right\rbrace
      $$
      
      The minimum of this function is defined at $m_1 = \frac{m}{2}$ and that gives the structure of complete binary tree.
  \end{itemize}