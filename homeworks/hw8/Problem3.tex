We've seen in class that we can create a Voronoi diagram in O($n$ $log$ $n$). From that we can imagine that before the algorithm we define an array of size $n$ keeping the closest point of $p_i$ at the $i$-th position. The idea is simple, we just want to keep track of the closest point for each $p_i$ while we contruct the Voronoi diagram. During the algorithm to build a Voronoi diagram every time we create an edge separating two points in $P$ we check for those two points if this new neighbor is closer to the one in the array. Checking and updating the array are basic operations so it can be done in O(1). Thus we just add some O(1) operations for the current steps in the original algorithm so the result of this new algorithm is still in O($n$ $log$ $n$).