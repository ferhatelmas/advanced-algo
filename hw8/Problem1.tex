\begin{enumerate}
  \item

  We will use binary search on the points so points must be sorted and $O(1)$ time access is preferable. Thus, our algorithm takes its input as a sorted array of points. 

  Since the polygon is convex, the angle to a reference point can increase but then it should decrease and if keeping going in the same way, after some time it will again increase. Let's define $a(x_i)$ is the angle $\angle{abx_i}$.

  \begin{itemize}
    \item Find the index $k$ of array where consecutive angles change sign. This is done by binary search with three angle calculations in each step of binary search.
    \begin{itemize}
      \item $sign(a(x_k) - a(x_{k-1})) = - sign(a_{k+1} - a_k)$
    \end{itemize}
    \item Note down $k$ because it is our tipping point where it will be used in the binary search for the intervals $[x_1, x_k]$ and $[x_k, x_n]$.
    \item if $a(x_k) - a(x_{k-1})$ is positive, $x_k$ is maximum point and then it starts to decrease. Otherwise, it is the minimum. According to $k$ being minimum or maximum, binary search only takes left or right on the given intervals.
    \item Binary search is run twice on the above two intervals to find the point that is doing maximum angle with function $a(x_i)$ as a comparison metric.
    \item Finally, we compare two local maximum angles to find the maximum and return the max of them. Noted that if global maximum is $k$, then we actually one angle because both binary search queries will get the same result.
  \end{itemize}

Analysis:
  \begin{itemize}
    \item We have used binary search to tipping point $x_k$ in $|H|$ points so the steps requires $\log(|H|)$ time.
    \item We had two intervals that are strictly less than $|H|$ so their complexity is also bounded by $\log(|H|)$.
    \item In total:
    \begin{align*}
      \textit{tipping point} + \textit{search in subintervals} &= O(\log(|H|) + 2 \cdot O(\log|H|) \\
      &= O(\log(|H|))   
    \end{align*}    
  \end{itemize}

  \item Analysis of the given algorithm:
    \begin{itemize}
      \item Algorithm returns $k$ points at the end and since these $k$ points define convex hull, $k$ equals to $s$.
      \item In the second loop of the algorithm, $k$ counts from 1 to $K$. $K$ is multiplied by itself in each step and when $K$ is bigger than $s$, algorithm ends. Therefore, it takes us $O(\log(s))$ steps to pass over $s$.
      \item In each step, we are spending $O(\frac{n}{K} \cdot K \log(K))$ time for convex hull calculations, $O(n)$ time for the point with the smallest y-coordinate and $O(\frac{n}{K} \cdot K \log(K))$ for the nested loop(first loop counts until $K$ and second loop counts until $m=\frac{n}{K}$ and inside of the loop is explained in the first part and it is $O(\log(K)$ since $|H_i|$ can be at most $K$).
      \item In total:
        \begin{align*}
          O(\log(s)) \cdot (2 \cdot(O(\frac{n}{K} \cdot K \log(K))) + n) &= O(\log(s)) \cdot O(n \cdot \log(K)) \\
          &= O(n \cdot \log(s)) \textit{ where } O(K) \leq O(s)
        \end{align*}               
    \end{itemize}
  

\end{enumerate}
