Firstly, we need to make a distinction between \textit{finite} and \textit{infinite} polygons because if we don't have any finite polygons, it is already Voronoi. Otherwise, we need to do the verification. Therefore, we choose an arbitrary finite polygon to find its generator point. We have a set of equations by the edges of polygon so that the point is inside of the polygon and intersection line groups perpendicular to the edges of the polygon. However, this equation group can give multiple solutions so we add equal distance equations into our group. This part can easily be solved by Gaussian elimination in $O(n^3)$ where $n$ is the number of the cells in the given Voronoi diagram.

After getting the first point, we easily get others in $O(n)$ by using equal-distance property. Every edge of our initial polygon is the median between the generator of its polygon and generators of the neighbour polygons. Therefore, we draw the perpendicular line to the edge from the found generator point and put a point, along the drawn line, which is as far from the edge as the initial generator.

Getting generator points enables us to switch into the dual of the problem, triangulation. We create Delaunay triangulation by connecting generators that takes $O(n)$ time since edges are around $O(n)$. Then, we check empty circle property for the each triangle and for each triangle, each vertex is traversed to check if it is inside the circumcircle of the triangle. Number of triangle is $O(n)$ and number of points is $O(n)$ so this step takes $O(n^2)$ time. We also need to check face property of the triangles because triangle pair must coincide in only one face or not at all. Therefore, we compare each triangle to others which can also be done in $O(n^2)$. Actually, it isn't a problem because they will be dominated due to the asymptotic behaviour of the Gaussian elimination($O(n^3)$). If these two tests pass, we return true. Otherwise, given subdivision isn't a Voronoi diagram.

Moreover, we could only traverse each polygon again to check how many points are contained in each polygon instead of going into triangulation. If there are no or multiple points(after intersection of the solution of the Gaussian system and generation via medians), we return false. Otherwise, we return true. However, overall complexity doesn't change and so it is just a matter of taste.

In short, Gaussian elimination sets the overall complexity which is $O(n^3)$.